% status: 0
% chapter: TBD


\title{OCR Technology Overview}


\author{Joao Paulo Leite}
\affiliation{%
\department{School of Informatics, Computing, and Engineering}
\institution{Indiana University}
\city{Bloomington}
\state{IN}
\postcode{47408}
\country{USA}}
\email{jleite@iu.edu}

\author{Gregor von Laszewski}
\affiliation{%
\institution{Indiana University}
\streetaddress{Smith Research Center}
\city{Bloomington} 
\state{IN} 
\postcode{47408}
\country{USA}}
\email{laszewski@gmail.com}


% The default list of authors is too long for headers}
\renewcommand{\shortauthors}{J. Leite}


\begin{abstract}

Optical Character Recognition (OCR) technology first appeared in the 1940's and grew alongside the rise of the digital computer. It was not until the late 1950's when OCR machines became commercially available and today this technology presents itself in both hardware devices as well as software offerings. At a high level, an OCR system consists of locating and segmenting each character, running the segmented character thru a pre-processor for normalization and noise reduction and extracting critical features to assist in the classification of each character. Once each character has been classified, the characters are regrouped and contextual information can be applied to assist in word construction and to detect potential character mis-classifcations. While OCR technology has continued to evolve over the years into the realms of handwriting recognition, known as Intelligent Character Recognition (ICR), the main issue with these systems have been around degraded characters, which are incorrectly fragmented or joined, which causes issues during the segmentation process. OCR technology has far-reaching applications and is typically the first step when attempting to provide automation to document-centric processes such as image classification and data entry/indexing. 

\end{abstract}

\keywords{hid-sp18-414, OCR, ICR, Optical Character Recognition, Computer Vision}


\maketitle


\section{Introduction}

The main principle in Optical Character Recognition (OCR) is to automatically recognize character patterns. This is done by teaching the system each class of pattern that can occur and providing a set of examples for each pattern. At the time of recognition, the system performs a comparison between the unknown character provided and the previously provided examples, assigned the appropriate class to the closest match. This system is designed to solely transform text on a document into machine encoded text and additional systems must be built to further extract relevant information from the document, that is to say, the process of OCR is the first step in transforming structured,semi-structured and unstructured documents into valuable and relevant information.

\section{Optical Character Recognition}
As stated in the name Optical Character Recognition, the characters that are typically trained are letter, numbers and special symbols. Each differing character is defined as its own class, and the system builds an understanding of each class utilizing examples of characters provided. The steps that will be performed are threshold processing, character segmentation, character preprocessing, feature extraction, classifcation and post processing. 


\subsection{Threshold Processing}

At it's core, the OCR process expects to process a black character which is presented against a white background. While images coming into an OCR system could have already undergone this transfromation from color image into a black and white image via a scanner, it is beneficial to perform this thersholding step before passing the image into the OCR engine to provide the high level image quality to the OCR engines. The mechanism behind this conversion is to analyse each pixel on the page to determine if it should be assigned as a black or white pixel. For color images that are on the RGB scale, this thresholding can be set at a fixed level so that any faintly colored pixels can be dropped as white while truly dark colored pixels are converted to black. In the case of grayscaled images, the same threshold can be set with the difference being the level of greyness presented in each pixel. Once this proccess is complete, the newly created black and white images will be used for the remainder of the process moving forward\cite{hid-sp18-414-www-imagethresholding}.

\subsection{Character Segmentation}

Character segmentation is a critical step in the process which represents breaking the image down into logical segments. While the system can be designed to segment the image into words, typically OCR is most successful if it is segmented to the lowest common denominator, the character. Each character is defined as a contiguously connected set of pixels and a break in the connection constitutes the beginning of a new character. While this may sound like a straightforward process, problems can occur when characters are fragmented or touching. Character distortions due to image quality issues or serifed font fonts are the main culprits behind fragmented or touching characters, while noise such as marks, handwriting and dots can also contribute to challenges when attempting to segment characters. To alleviate this issue, before the characters are presented to the feature extraction phase in the process, the characters are run thru the preprocessing phase in an attempt to correct some of the issues that may have manifested themselves\cite{hid-sp18-414-www-eikvilocr}.

\subsection{ Character Preprocessing}

Character Preprocessing is a vital step that occurs before the extraction/classification, with the goal being to provide the best quality character to subsequent steps. A certain amount of character defects can be introduced during the scan process as well as the thresholding step, which can later cause poor character level recognition rates.

To combat these defects, a preprocessor is employed to attempt to correct these issues and a common technique is called smoothing. Smoothing serves to both fill in gaps within a character (fragmentation correction) as well as thin the width of lines within a character (touching correction). When properly applied, smoothing is successful in filling in pits within a character and removing bumps from a character, which will increase the likelihood of recognition in the following steps\cite{hid-sp18-414-www-eikvilocr}. Removing noise and normalization of the character are also considered tasks, which will be resolved by the preprocessor. The removal of specks, thin lines and other inconsistency are resolved thru the analysis of the height, size and density of a grouping of pixels. If the characteristics of a particular grouping is not consistent with the characteristics found for a typical character, the grouping is deemed noised and removed as such. The normalization of characters is applied to provide a uniformly sized and oriented character, fixing issues around scaling, slanting and rotation of characters. 

\subsection{Feature Extraction}

The simplest extraction technique, template matching, foregoes feature analysis and will only compare the inputted character against a known set of characters provided for each class. The distance between the inputted character and the set of known characters is computed for each class. Once that comparison is completed, the class with lowest distance is assigned as the class for the inputted character. While this method is simple, the simplicity does not afford any flexibility around noise or font variations, which have not been assigned. 

Because of rigidity of the template matching technique, feature based techniques were developed to extract significant features from a character. Some common feature extraction methods are zoning, distance profiling and directional distribution analysis\cite{hid-sp18-414-www-featureextraction}.
\subsubsection{Zoning}
Zoning is a technique that frames the character in a set of overlapping or non-overlapping zones. The pixel density in each zone must be calculated by taking the number of black pixels in the zone divided by the total number of pixels presented in the zone.

\subsubsection{Distance Profiling}
Distance profiling is a technique that frames the character in a bounding box. The distance from the bounding box to the outer edge of the character is calculated for each of the four side (top, bottom, left and right).

\subsubsection{Directional Distribution}
Directional distribution analysis is a technique that assigned a center point to the character. Once the center point is assigned, the weight is calculated by taking the number of black pixels found in each direction divided by the total number of pixels found in the character.

Because these techniques are independent, there are possibilities to combine multiple features to increase the accuracy of recognition. 

\subsection{Classification}

The classification step is the culmination of all the previous steps to obtain the desired result of assigning a character to the correct class. One such classification method that could be used is K- Nearest Neighbor.

\subsubsection{K-Nearest Neighbor}
The K-Nearest Neighbor (k-NN) provides a method to classify characters based on the closest features extracted in the training examples. Typically regarded as a simple machine learning algorithm, k-NN calculates the Euclidean distance between features value of the inputted character against the features value of the characters provided by the training examples. Once the distance is calculated, the results are arranged in order and the input character is assigned the character class that corresponds to the majority of its nearest neighbors\cite{hid-sp18-414-www-featureextraction}.


\subsection{Post Processing}


\subsubsection{Grouping}

Once all the individual characters have been successfully classified, the system can begin to group those set of characters into the next level of association. Grouping characters into logical strings of words, numbers or tokens is an easy task of considering the location of each individual character and evaluating the pixel distance (white space) to the next individual character. With machine printed text, the assumption is that distances between words are far greater than distances between characters within a word. 

\subsubsection{Error-Detection}

Because individual character recognition will never be 100 percent accurate, we can utilize the context around our newly formed words from the grouping phase to increase the accuracy and detect errors around the recognition. This secondary evaluation process will be based on the systems understanding of the underlying language for which the text is written in.

\subsubsection{Language Syntax}
One form to evaluate the accuracy is to use the syntax of the language and rule out specific combinations of characters appearing in sequence. As an example, if the recognition for the three-letter word "cut" came back as "cwt", the system would understand that the syntax of a C followed by a W and a W followed by a T is highly improbable in the English language and flag this a potential error\cite{hid-sp18-414-www-eikvilocr}. 

\subsubsection{Dictionaries}
Another evaluation method that can assist with the accuracy is a dictionary lookup. Following the logic of the example above, after understanding that we have mistakenly extracted "cwt", we can apply dictionaries to assist in correcting the error that was caused by the individual character recognition engine. Because "w" and "u" share some common characteristics, the original classification can be utilized to not only provide the highest matching character but also consider which matching characters provides the highest probability of forming a word that matches an entry in the dictionary\cite{hid-sp18-414-www-eikvilocr}. 

\section{Conclusion}




\begin{acks}

The author would like to thank Dr.~Gregor~von~Laszewski for his
support and suggestions to write this paper.

\end{acks}

\bibliographystyle{ACM-Reference-Format}
\bibliography{report} 

s