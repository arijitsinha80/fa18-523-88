% status: 100
% chapter: Optical Characte Recognition

\title{OCR Extraction Implementation with Tesseract}


\author{Joao Paulo Leite}
\affiliation{%
\institution{Indiana University}
\streetaddress{Smith Research Center}
\city{Bloomington} 
\state{IN} 
\postcode{47408}
\country{USA}}
\email{jleite@ui.com}

% The default list of authors is too long for headers}
\renewcommand{\shortauthors}{J. P. Leite}


\begin{abstract}

The main purpose of this paper is to create a simple OCR extraction
implementation which is able to extract key metadata from documents.
To accomplish this, Google's Tesseract OCR Engine is leveraged to provide
full-page OCR data. The goal is to have a configurable extraction engine 
that allows users to pin-point the meta-data to be extracted and
output said meta-data.

\end{abstract}

\keywords{hid-sp18-414, OCR, Tesseract, Python}


\maketitle

\section{Introduction}

Optical Character Recognition (OCR) technology first appeared in the 1940's
and grew alongside the rise of the digital computer. It was not until the late 
1950's when OCR machines became commercially available and today this 
technology presents itself in both hardware devices as well as software offerings
\cite{hid-sp18-414-www-eikvilocr}.OCR is the first step in enabling the extraction 
of actionable data by transforming print on an image(document) to machine encoded 
text. The analysis of the output provided by OCR engines allows for this key data to 
be used for downstream processes and reporting. Documents fall into three categories: 
structured documents, semi-structured documents and unstructured documents. Gartner, 
a leading technology analysis firm, has stated the following, “…the amount of data stored
in companies will increase by 800 percent by 2018, 80 percent of which would include 
unstructured data that are harder to tame and manage. The biggest challenges for companies
will include: collecting, managing, storing, searching and archiving this content
\cite{hid-sp18-414-www-ecmandbigdata}.” As unstructured documents 
continues to grow, big data systems are being introduced as a solution to analyze and 
organize this data. As a precursor, an OCR extraction solution can extract actionable
data from documents and provide structure to unstructured content. 

\section{Overview of Optical Character Recognition}

The main principle in Optical Character Recognition (OCR) is to 
automatically recognize character patterns. This is accomplished by teaching 
the system each class of pattern that can occur and providing a set of 
examples for each pattern. At the time of recognition, the system 
performs a comparison between the unknown character provided and 
the previously provided examples, assigned the appropriate class to 
the closest match \cite{hid-sp18-414-www-eikvilocr}. This system is 
designed to solely transform text on a document into machine encoded 
text and additional systems must be built to further extract relevant information
from the document, that is to say, the process of OCR is the first step 
in transforming structured, semi-structured and unstructured documents 
into valuable and relevant information.

\section{Context Based Extraction Engine}

This project utilizes Google's Open Source Tesseract OCR engine to provide
HOCR output that is leveraged to begin the process of extracting information 
from unstructured data provided by Tesseract. The extraction engine's logic works 
in two distinct phases, the identification of potential candidates( data which follows 
a specific format) and the scoring of each candidate based on context around said 
candidate. At the end of this process, the candidate which obtained the highest 
score will be selected. 

\subsection{Image Thresholding}

Before submitting the image into Tesseract, image clean
up is performed to create a bitonal image and to remove any noise that
may be present. This process consists of three steps; standardizing 
image DPI, smoothing the image and removing noise from the
 image\cite{hid-sp18-414-www-imagethresholding}.

\bigskip
\noindent
\textbf{Standarizing Image DPI to 300 DPI}
\begin{footnotesize}
\begin{verbatim}

def set_dpi(path):

image = IMG.open(path)
len_x, wid_y = image.size
factor = max(1, int(1800 / len_x))
size = factor * len_x, factor * wid_y
image_resized = image.resize(size, IMG.ANTIALIAS)
temp_f = tempfile.NamedTemporaryFile()
temp_fn = temp_f.name
image_resized.save(temp_fn, dpi=(300, 300))

return temp_fn

\end{verbatim}
\end{footnotesize}

\bigskip
\noindent
\textbf{Converting to Bitonal Image via Adaptive Thresholding}
\begin{footnotesize}
\begin{verbatim}

def remove_noise(name):
image = cv2.imread(name, 0)
filtered = cv2.adaptiveThreshold(image.astype(np.uint8), 255, 
cv2.ADAPTIVE_THRESH_MEAN_C, cv2.THRESH_BINARY, 41, 3)
core = np.ones((1, 1), np.uint8)
opening = cv2.morphologyEx(filtered, cv2.MORPH_OPEN, core)
closing = cv2.morphologyEx(opening, cv2.MORPH_CLOSE, core)
image = smooth(image)
original_image = cv2.bitwise_or(image, closing)
return original_image

return False
\end{verbatim}
\end{footnotesize}


\bigskip
\noindent
\textbf{Smoothing Image}
\begin{footnotesize}
\begin{verbatim}

def smooth(image):

ret1, th1 = cv2.threshold(image, BINARY_THREHOLD, 255, cv2.THRESH_BINARY)
ret2, th2 = cv2.threshold(th1, 0, 255, cv2.THRESH_BINARY + cv2.THRESH_OTSU)
blur = cv2.GaussianBlur(th2, (1, 1), 0)
ret3, th3 = cv2.threshold(blur, 0, 255, cv2.THRESH_BINARY + cv2.THRESH_OTSU)
return th3

\end{verbatim}
\end{footnotesize}
\bigskip
\noindent

\subsection{Tesseract HOCR Process}

Google's Tesseract OCR engine is an open source engine that has the ability to
output HOCR. HOCR is an open standard of data representation for
formatted text obtained from an OCR engine. This standard includes 
text, style, layout information, recognition confidence and other info
in a XML structure. 

\bigskip
\noindent
\textbf{Create HOCR data:}
\begin{footnotesize}
\begin{verbatim}

def Run(self):
DATA = pytesseract.image_to_pdf_or_hocr
(image, lang=None, config='hocr', nice=0, extension='hocr')

\end{verbatim}
\end{footnotesize}


\subsection{Transform HOCR Data}

Once the HOCR results are generated, we must transform the results
into useable data for our extraction process. The first step is to target
the ocrx_word data from the results and parser it into separate words.
After this initial parsing is complete, we separate each individual data
point within a dictionary object with the values: value, confidence, 
left, top, right and bottom.

\bigskip
\noindent
\textbf{Parsing HOCR results with Beautiful Soup}
\begin{footnotesize}
\begin{verbatim}

soup = bs4.BeautifulSoup(DATA, 'html.parser')
words = soup.find_all('span', class_='ocrx_word')

\end{verbatim}
\end{footnotesize}


\bigskip
\noindent
\textbf{Creating word data structure}
\begin{footnotesize}
\begin{verbatim}

def transform_hocr(self, words):
# Convert HOCR to usable structure
for x in range(len(words)):
word[int(words[x]['id'].split('_')[2])] = {}
word[int(words[x]['id'].split('_')[2])]['Value'] = words[x].get_text()
word[int(words[x]['id'].split('_')[2])]['Confidence'] = words[x]['title'].split(';')[1].split(' ')[2]
word[int(words[x]['id'].split('_')[2])]['Left'] = words[x]['title'].split(';')[0].split(' ')[1]
word[int(words[x]['id'].split('_')[2])]['Top'] = words[x]['title'].split(';')[0].split(' ')[2]
word[int(words[x]['id'].split('_')[2])]['Right'] = words[x]['title'].split(';')[0].split(' ')[3]
word[int(words[x]['id'].split('_')[2])]['Bottom'] = words[x]['title'].split(';')[0].split(' ')[4]


\end{verbatim}
\end{footnotesize}

\subsection{Define Candidates}

After transforming the HOCR results, we use the generated word dictionary 
to find values that match the defined regular expression that was provided
by the user. We store all candidates which match the regular expression are 
stored within the candidates dictionary object with the values: value, confidence, 
left, top, right and bottom.

\bigskip
\noindent
\textbf{Finding candidates:}
\begin{footnotesize}
\begin{verbatim}

def find_candidates(self, RE_ATT):
y = 1
for z in RE_ATT:

for x in range(len(word)):

m = re.match(r'' + z + '', word[x + 1]['Value'], )

if m:
candidates[y] = {}
candidates[y]['Value'] = word[x + 1]['Value']
candidates[y]['Confidence'] = word[x + 1]['Confidence']
candidates[y]['Left'] = word[x + 1]['Left']
candidates[y]['Top'] = word[x + 1]['Top']
candidates[y]['Right'] = word[x + 1]['Right']
candidates[y]['Bottom'] = word[x + 1]['Bottom']
y = y + 1
\end{verbatim}
\end{footnotesize}

\subsection{Set Context}

Using the location input define by the user, we will set the context of each
candidate based on the proximity(top, bottom, left and right) in pixels.
Each word which falls within the proper proximity is stored in the context
dictionary with the values: value, candidate ,word number, confidence,
left, top, right, bottom, line number and same line as candidate.

\bigskip
\noindent
\textbf{Finding candidates:}
\begin{footnotesize}
\begin{verbatim}

def set_context(self, candidates, word):
line = 1
z = 1
for x in range(len(candidates)):

for y in range(len(word)):

if (int(word[y + 1]['Bottom']) > int(candidates[x + 1]['Bottom']) - 100) and (
int(word[y + 1]['Bottom']) < int(candidates[x + 1]['Bottom']) + 20) and \
(int(word[y + 1]['Right']) > int(candidates[x + 1]['Left']) - 700) and (
int(word[y + 1]['Right']) < int(candidates[x + 1]['Left']) + 20):

context[z] = {}
context[z]['Value'] = word[y + 1]['Value']
context[z]['Candidates'] = candidates[x + 1]['Value']
context[z]['Word'] = str(y + 1)
context[z]['Confidence'] = word[y + 1]['Confidence']
context[z]['Left'] = word[y + 1]['Left']
context[z]['Top'] = word[y + 1]['Top']
context[z]['Right'] = word[y + 1]['Right']
context[z]['Bottom'] = word[y + 1]['Bottom']

if z == 1:
context[z]['Line'] = line
elif context[z - 1]['Bottom'] == word[y + 1]['Bottom']:
context[z]['Line'] = line
else:
line = line + 1
context[z]['Line'] = line

if int(word[y + 1]['Bottom']) > int(candidates[x + 1]['Bottom']) - 15 and int(
word[y + 1]['Bottom']) < int(candidates[x + 1]['Bottom']) + 15:
context[z]['SameLine'] = "1"
else:
context[z]['SameLine'] = "0"

z = z + 1

\end{verbatim}
\end{footnotesize}

\subsection{Group Context}

Once the context for each candidate has been
defined, we will group the context based on proximity
If mutliple context words are in sequence, we wil group 
those so that they are arranged as a phrase.

\bigskip
\noindent
\textbf{Grouping Context:}
\begin{footnotesize}
\begin{verbatim}

def define_groupcontext(self, context):
# TRANSFORM CONTEXT INTO GROUPED CONTEXT
# Context words that are on the same line and in sequence are grouped together
z = 1
for x in range(len(context)):

if x == 0:
groupcontext[z] = {}
groupcontext[z]['Value'] = context[x + 1]['Value']
groupcontext[z]['Word'] = context[x + 1]['Word']
groupcontext[z]['Candidates'] = context[x + 1]['Candidates']
groupcontext[z]['Weight'] = '0'
groupcontext[z]['Confidence'] = context[x + 1]['Confidence']
groupcontext[z]['Left'] = context[x + 1]['Left']
groupcontext[z]['Top'] = context[x + 1]['Top']
groupcontext[z]['Right'] = context[x + 1]['Right']
groupcontext[z]['Bottom'] = context[x + 1]['Bottom']
groupcontext[z]['SameLine'] = context[x + 1]['SameLine']

elif int(groupcontext[z]['Word']) + 1 == int(context[x + 1]['Word']):

groupcontext[z]['Value'] = groupcontext[z]['Value'] + ' ' + context[x + 1]['Value']
groupcontext[z]['Word'] = context[x + 1]['Word']
groupcontext[z]['Confidence'] = context[x + 1]['Confidence']
groupcontext[z]['Top'] = context[x + 1]['Top']
groupcontext[z]['Right'] = context[x + 1]['Right']
groupcontext[z]['Bottom'] = context[x + 1]['Bottom']

else:
z = z + 1
groupcontext[z] = {}
groupcontext[z]['Value'] = context[x + 1]['Value']
groupcontext[z]['Word'] = context[x + 1]['Word']
groupcontext[z]['Candidates'] = context[x + 1]['Candidates']
groupcontext[z]['Weight'] = '0'
groupcontext[z]['Confidence'] = context[x + 1]['Confidence']
groupcontext[z]['Left'] = context[x + 1]['Left']
groupcontext[z]['Top'] = context[x + 1]['Top']
groupcontext[z]['Right'] = context[x + 1]['Right']
groupcontext[z]['Bottom'] = context[x + 1]['Bottom']
groupcontext[z]['SameLine'] = context[x + 1]['SameLine']

\end{verbatim}
\end{footnotesize}

\subsection{Score Context}

After grouping the context, using the context values provided by
the user, we will score each grouping based on how strongly it matches 
the context values. We utilize a fuzzy algorithm that allows us to accommodate
for any OCR errors or misspellings. The weight given to each context word is also
judged based on the weighted value provided by the user. This gives the ability for
the user to define which context words should carry more weight in the scoring
algorithm. For grouped context that fall within the same line as the candidate, the user
can define a value to be added to the overall weight. 

\bigskip
\noindent
\textbf{Score Context:}
\begin{footnotesize}
\begin{verbatim}


def weightcontext(self, KW_ATT):
# Match Context and Weighting

for z, value in KW_ATT.items():

for x in range(len(groupcontext)):

groupcontext[x + 1]['Weight'] = 0

if int(groupcontext[x + 1]['Weight']) < fuzz.WRatio(groupcontext[x + 1]['Value'], z):
groupcontext[x + 1]['Weight'] = int(fuzz.WRatio(groupcontext[x + 1]['Value'], z)) * int(
value[0]) / 100

if groupcontext[x + 1]['SameLine'] == '1':
groupcontext[x + 1]['Weight'] = groupcontext[x + 1]['Weight'] + int(value[5])

\end{verbatim}
\end{footnotesize}

\subsection{Output Results}

Outputting a resulting text file with the winning candidate as well as
the entire results array. The text file name will be the same as the input 
image file.

\bigskip
\noindent
\textbf{Score Context:}
\begin{footnotesize}
\begin{verbatim}

def outputresults(self, groupcontext,fp):
# Output Results
for x in range(len(groupcontext)):

if groupcontext[x + 1]['Candidates'] in results:

if int(results[groupcontext[x + 1]['Candidates']])/
< int(groupcontext[x + 1]['Weight']):
results[groupcontext[x + 1]['Candidates']] =/
groupcontext[x + 1]['Weight']

else:
results[groupcontext[x + 1]['Candidates']] =/ 
groupcontext[x + 1]['Weight']


if(len(results.keys()) == 0):

f = open(fp + '.txt', 'w')
f.write("Could not find any valid candidates")
f.close()

else:
sorted_by_value = sorted(results.items(), key=/
lambda kv: kv[1], reverse=True)
f = open(fp +'.txt', 'w')
f.write("WINNING CANDIDATE (CANDIDATE , WEIGHT): " 
+ str(sorted_by_value[0]) + "\n")
f.write("ALL CANDIDATES: " + str(sorted_by_value))
f.close()
\end{verbatim}
\end{footnotesize}


\section{Examples}

\subsection{Finding meta-data from semi-structured documents}

\bigskip
\noindent
\begin{footnotesize}
\begin{verbatim}

request:
curl -H "Content-Type: application/json" \
-X POST \
-d '{"nodes": ["http://127.0.0.1:8888"]}' \
http\://localhost\:9999/register

response:
{
"message": "Nodes Added",
"nodes": [
"127.0.0.1:8888"
]
}

\end{verbatim}
\end{footnotesize}


\section{Tools and Technology}

The tools and technology deployed for this project are going to be covered in 
this section.

\subsection{Terresact}

Python-tesseract is an optical character recognition (OCR) tool for python.
That is, it will recognize and “read” the text embedded in images. Python-tesseract 
is a wrapper for Google’s Tesseract-OCR Engine\cite{hid-sp18-414-www-pytesseract}.

\bigskip
\noindent
\textbf{Code Example:}
\begin{footnotesize}
\begin{verbatim}

import pytesseract

# get HOCR output
hocr = pytesseract.image_to_pdf_or_hocr('test.png', extension='hocr')

\end{verbatim}
\end{footnotesize}
\noindent
\textbf{Install:}
\begin{footnotesize}
\begin{verbatim}

pip install pytesseract
\end{verbatim}
\end{footnotesize}

\subsection{Beautiful Soup}

Beautiful Soup is a library that makes it easy to scrape information from web pages.
It sits atop an HTML or XML parser, providing Pythonic idioms for iterating, searching,
and modifying the parse tree\cite{hid-sp18-414-www-beautifulsoup}.

\bigskip
\noindent
\textbf{Code Example:}
\begin{footnotesize}
\begin{verbatim}

import bs4
soup = bs4.BeautifulSoup(DATA, 'html.parser')
words = soup.find_all('span', class_='ocrx_word')

\end{verbatim}
\end{footnotesize}
\noindent
\textbf{Install:}
\begin{footnotesize}
\begin{verbatim}

pip install beautifulsoup4
\end{verbatim}
\end{footnotesize}


\subsection{FuzzyWuzzy}

Fuzzy Wuzzy provides fuzzy string matching in an easy to use package.
It uses Levenshtein Distance to calculate the differences between sequences
in a simple-to-use package\cite{hid-sp18-414-www-fuzzywuzzy}.


\bigskip
\noindent
\textbf{Code Example:}
\begin{footnotesize}
\begin{verbatim}

from fuzzywuzzy import fuzz

fuzz.ratio("fuzzy wuzzy was a bear", "wuzzy fuzzy was a bear")
OUTPUT:91

\end{verbatim}
\end{footnotesize}
\noindent
\textbf{Install:}
\begin{footnotesize}
\begin{verbatim}

pip install fuzzywuzzy
\end{verbatim}
\end{footnotesize}

\subsection{Python}

Python is the high-level programming language that was used to develop
this project.

\subsection{numpy}

NumPy is the fundamental package for scientific computing with Python\cite{hid-sp18-414-www-NumPy} 

\bigskip
\noindent
\textbf{Code Example:}
\begin{footnotesize}
\begin{verbatim}
import numpy as np
core = np.ones((1, 1), np.uint8)
\end{verbatim}
\end{footnotesize}
\noindent
\textbf{Install:}
\begin{footnotesize}
\begin{verbatim}
pip install Numpy
\end{verbatim}
\end{footnotesize}

\subsection{OpenCV}

OpenCV (Open Source Computer Vision Library) is released under a BSD license and hence it’s
free for both academic and commercial use. It has C++, Python and Java interfaces and
supports Windows, Linux, Mac OS, iOS and Android. OpenCV was designed for computational 
efficiency and with a strong focus on real-time applications.\cite{hid-sp18-414-www-OpenCV}

https://opencv.org/

\bigskip
\noindent
\textbf{Code Example:}
\begin{footnotesize}
\begin{verbatim}

import cv2

# Load an color image in grayscale
img = cv2.imread('messi5.jpg',0)

\end{verbatim}
\end{footnotesize}
\noindent
\textbf{Install:}
\begin{footnotesize}
\begin{verbatim}
pip install opencv-python
\end{verbatim}
\end{footnotesize}

\subsection{Python Imaging Library}

The Python Imaging Library adds image processing capabilities 
to your Python interpreter. This library provides extensive file 
format support, an efficient internal representation, and fairly 
powerful image processing capabilities.\cite{hid-sp18-414-www-Pillow}


\bigskip
\noindent
\textbf{Code Example:}
\begin{footnotesize}
\begin{verbatim}

from PIL import Image as IMG

image = IMG.open(path)

\end{verbatim}
\end{footnotesize}
\noindent
\textbf{Install:}
\begin{footnotesize}
\begin{verbatim}
pip install Pillow
\end{verbatim}
\end{footnotesize}


\subsection{Tkinter}

Tkinter is Python's a standard GUI (Graphical User Interface) package.
It is a thin object-oriented layer on top of Tcl/Tk\cite{hid-sp18-414-www-tkInter}.

\bigskip
\noindent
\textbf{Code Example:}
\begin{footnotesize}
\begin{verbatim}
nodeid = str(uuid1()).replace('-', "")
\end{verbatim}
\end{footnotesize}
\noindent
\textbf{Install:}
\begin{footnotesize}
\begin{verbatim}
pip install uuid
\end{verbatim}
\end{footnotesize}

\section{Conclusion}


\begin{acks}

The authors would like to thank Dr.~Gregor~von~Laszewski for his
support and suggestions to write this paper.

\end{acks}

\bibliographystyle{ACM-Reference-Format}
\bibliography{report} 


